\documentclass{article}
\usepackage[utf8]{inputenc}
\usepackage[spanish]{babel}
\usepackage{listings}
\usepackage{graphicx}
\graphicspath{ {images/} }
\usepackage{cite}

\begin{document}

\begin{titlepage}
    \begin{center}
        \vspace*{1cm}
            
        \Huge
        \textbf{Parcial 1}
            
        \vspace{0.5cm}
        \LARGE
        Calistenia
            
        \vspace{1.5cm}
            
        \textbf{María Daniela Rodríguez Chacón}
            
        \vfill
            
        \vspace{0.8cm}
            
        \Large
        Despartamento de Ingeniería Electrónica y Telecomunicaciones\\
        Universidad de Antioquia\\
        Medellín\\
        Marzo de 2021
            
    \end{center}
\end{titlepage}

\tableofcontents
\newpage
\section{\textit{Introducción}}\label{intro}
Seguir instrucciones es una forma básica de lógica que lleva a la sociedad a tener orden en los distintos procesos cotidianos, por ello, se procedió a indicarle una serie de pasos a un grupo de personas para verificar de cuántas maneras distintas pueden percibir una información. 

\section{Instrucciones a seguir} \label{contenido}
Las siguientes instrucciones fueron entregadas a las tres personas que hicieron parte del proyecto:
\begin{enumerate}
    \item Colocar una hoja de papel blanco sobre dos tarjetas plásticas del mismo tamaño (una sobre la otra de forma vertical) encima de una mesa.
    \item Tomar la hoja de papel y colocarla sobre la mesa junto a las tarjetas.
    \item Tomar ambas tarjetas con una sola mano y acostarlas sobre la hoja de papel verticalmente.
    \item Colocar el dedo pulgar en la parte superior izquierda de las tarjetas y el dedo índice en la parte superior derecha de las tarjetas.
    \item Levantar las tarjetas y sostenerlas de forma vertical.
    \item Sin mover el dedo pulgar e índice, colocar el dedo anular y/0 meñique en la parte lateral inferior izquierda de las tarjetas.
    \item Separar poco a poco la parte inferior de las tarjetas hasta formar una pirámide.
    \item Separar lentamente los dedos pulgar e índice de las tarjetas.
\end{enumerate}

\newpage
\section{Inclusión de imágenes} \label{imagenes}

En la Figura (\ref{fig:Captura}), se representa la posición inicial de los objetos.

\begin{figure}[h]
\includegraphics[width=4cm]{Captura1.PNG}
\centering
\caption{Posición inicial}
\label{fig:Captura}
\end{figure}

En la Figura (\ref{fig:Captura}), se representa la posición final de los objetos.

\begin{figure}[h]
\includegraphics[width=4cm]{Captura.PNG}
\centering
\caption{Posición final}
\label{fig:Captura}
\end{figure}

\end{document}
